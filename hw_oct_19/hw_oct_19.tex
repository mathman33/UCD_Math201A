\documentclass[12pt]{article}
\textwidth=17cm \oddsidemargin=-0.9cm \evensidemargin=-0.9cm
\textheight=23.7cm \topmargin=-1.7cm

\usepackage{amssymb, amsmath, amsfonts}
\usepackage{moreverb}
\usepackage{graphicx}
\usepackage{enumerate}
\usepackage{graphics}
\usepackage{color}
\usepackage{array}
\usepackage{float}
\usepackage{hyperref}
\usepackage{textcomp}
\usepackage{alltt}
\usepackage{physics}
\usepackage{mathtools}
\usepackage{tikz}
\usetikzlibrary{positioning}
\usetikzlibrary{arrows}
\usepackage{pgfplots}
\usepackage{bigints}
\allowdisplaybreaks

\newcommand{\suchthat}{\, \mid \,}
\renewcommand{\theenumi}{\alph{enumi}}
\newcommand\Wider[2][3em]{%
\makebox[\linewidth][c]{%
  \begin{minipage}{\dimexpr\textwidth+#1\relax}
  \raggedright#2
  \end{minipage}%
  }%
}

\setcounter{section}{-1}

\title{\bf HW \#1}
\author{\bf Sam Fleischer}
\date{\bf October 19, 2015}

\begin{document}
{\bf MATH 201A \hfill Applied Analysis \ \ \ \ \ \hfill Fall 2015} 

{\let\newpage\relax\maketitle}

\section*{Problem 1}
\emph{Let $(f_n)$ be a sequence in $C([0,1])$ converging uniformly to the function $f(x) = -x\log x$ on $[0,1]$.  Define}
\begin{align*}
	A = \{f_n\ |\ n \geq 1\}\cup\{f\}
\end{align*}
\emph{Is A compact, or precompact but not compact, or not precompact, considered as a subset of $(C([0,1]), \norm{\cdot}_{\text{sup}})$?  Justify your answer.} \\

Let $G = \{G_i\ |\ i\in I\}$ for some index set $I$ be an open cover of $A$.  Then $\exists i_0 \in I$ such that $f \in G_{i_0}$.  Since $G_{i_0}$ is open, $\exists \epsilon > 0$ such that $B_\epsilon(f) \subset G_{i_0}$.  Since $f_n$ converges uniformly to $f$, $\exists N\in\mathbb{N}$ such that $n \geq N \implies f_n\in B_\epsilon(f)$ (and thus $f_n \in G_{i_0}$).  Then there are only finitely many functions in $A$ which are potentially not elements of $B_\epsilon(f)$, specifically, $f_1,\dots, f_{N-1}$.  Since $G$ is an open cover of $A$, $\exists i_{1}, \dots, i_{N-1} \in I$ such that $f_1 \in G_{i_1}, \dots, f_{N-1} \in G_{i_{N-1}}$.  Thus $\widetilde{G} = \{G_{i_0}, G_{i_1}, \dots, G_{i_{N-1}}\}$ is a finite open cover of $A$.  Thus $A$ is compact. \hfill $\square$

\section*{Problem 2}
\emph{Let $f \in C([a,b])$.  Prove that}
\begin{equation*}
	\left|\int_a^b f(x) \dd x\right| \leq |b - a|^{1/2}\qty(\int_a^b f(x)^2 \dd x)^{1/2}
\end{equation*} \\

Define the inner product $<f,g>$ on $C([a,b])$ to be the $\textbf{\emph{L}}^2$ inner product, or
\begin{align*}
	<f,g> = \int_a^b f(x) g(x) \dd x
\end{align*}
Then consider Cauchy-Schwarz inequality: $|<f,g>| \leq \norm{f}\cdot\norm{g}$, $\forall f, g \in C([a,b])$.  Pick $g(x) \equiv 1$.  Then,
\begin{align*}
	|<f,1>| &= \left|\int_a^b f(x) \dd x\right|\\
	&\leq \qty(\int_a^b f(x)^2\dd x)^{1/2}\qty(\int_a^b 1^2\ \dd x)^{1/2}\ \ \ \ \ \text{by the Cauchy-Schwarz inequality} \\
	&= \qty(x\Big|_a^b)^{1/2} \qty(\int_a^b f(x)^2\dd x)^{1/2}\\
	&= |b - a|^{1/2}\qty(\int_a^b f(x)^2\dd x)^{1/2}\ \ \ \ \ \text{$b - a = |b - a|$ since $b \geq a$.}
\end{align*}\hfill $\square$

\section*{Problem 3}
\emph{For $M > 0$, define $A_M \subset C([a,b])$ as follows:}
\begin{equation*}
	A_M = \{f \in C([a,b])\ |\ f' \in C([a,b]), f(a) = f(b) = 0, \text{ and } \int_a^b f'(x)^2 \dd x \leq M\}
\end{equation*}
\emph{Prove that $A_M$ is precompact in $(C([a,b]), \norm{\cdot}_{\text{sup}})$.} \\

Let $\tilde{x} \in [a,b]$ and $f \in A_M$.  Then,
\begin{align*}
	|f(\tilde{x})| = |f(\tilde{x}) - 0| &= |f(\tilde{x}) - f(a)| \\
	&= \left|\int_a^{\tilde{x}} f'(x)\dd x\right|\ \ \ \ \ \text{by the Fundamental Theorem of Calculus} \\
	&\leq |\tilde{x} - a|^{1/2}\qty(\int_a^{\tilde{x}} f'(x)^2 \dd x)^{1/2} \ \ \ \ \ \text{by Problem 2} \\
	&\leq |b - a|^{1/2}\qty(\int_a^{\tilde{x}} f'(x)^2 \dd x)^{1/2}\ \ \ \ \ \text{since $\tilde{x}\in [a,b]$}\\
	&\leq |b - a|^{1/2}\sqrt{M}\ \ \ \ \ \text{by the definition of $A_M$}
\end{align*}
Thus $f(\tilde{x})$ is uniformly bounded by $|b-a|^{1/2}\sqrt{M}$ for any $\tilde{x} \in [a,b]$ and any $f \in A_M$.  Thus $A_M$ is bounded.

Pick $\epsilon > 0$ and $x_1 \in [a,b]$.  Assume $d(x_1, x_2) < \frac{\epsilon^2}{M}$.  Then
\begin{align*}
	|d(f(x_1), f(x_2))| &= |f(x_1) - f(x_2)| \\
	&= \left|\int_{x_2}^{x_1}f'(x) \dd x\right|\ \ \ \ \ \text{by the Fundamental Theorem of Calculus} \\
	&\leq |x_2 - x_1|^{1/2}\qty(\int_{x_2}^{x_1} f'(x)^2)^{1/2}\ \ \ \ \ \text{by Problem 2} \\
	&< \sqrt{\frac{\epsilon^2}{M}}\sqrt{M}\ \ \ \ \ \text{by assumption and the definition of $A_M$}\\
	&= \epsilon
\end{align*}
Thus $A_M$ is equicontinuous.  By the Arzel\`a-Ascoli Theorem, $A_M$ is precompact. \hfill $\square$

\section*{Problem 4}
\emph{Consider functions $f: [0,1] \rightarrow \mathbb{R}$ defined by}
\begin{equation}
	\label{form_1}
	f(x) = \sum_{n=1}^\infty a_n\sin(n\pi x),\ \ x \in [0,1]
\end{equation}
\emph{where for all $n \geq 1$, $a_n \in \mathbb{R}$, and such that $\sum_{n=1}^\infty |a_n| < +\infty$.}

\subsection*{ a)}
\emph{Prove that $f \in C([0,1])$.} \\

Consider the sequence $f_k(x) = \sum_{n=1}^k a_n \sin(n\pi x)$.  Clearly, $\lim_{k\rightarrow\infty}f_k = \sum_{n=1}^\infty a_n \sin (n\pi x) = f(x)$.  Since $\sin(n\pi x) \in C([0,1])$ for $n = 1, 2, \dots$, and linear combinations of continuous functions are continuous, $f_k \in C([0,1])$ for $k = 1, 2, \dots$.  It suffices to show that $f_k$ is a Cauchy sequence.  Then, the completeness of $C([0,1])$ will imply that the limit of $f_k$ is in $C([0,1])$.

First, since $\lim_{k\rightarrow\infty}\sum_{n=1}^k|a_n| = L < \infty$, then $\forall \epsilon>0$, $\exists N\in\mathbb{N}$ such that $i > j \geq N \implies \sum_{n=1}^i |a_n| - \sum_{n=1}^j |a_n| = \sum_{n=j+1}^i |a_n| < \epsilon$.  Now pick $\epsilon > 0$, and assume $i > j \geq N$, such that $\sum_{n=j+1}^i |a_n| < \epsilon$.  Then,
\begin{align*}
	d(f_i, f_j) = \norm{f_i - f_j}_{\text{sup}} &= \norm{\sum_{n=j+1}^i a_n sin(n\pi x)}_{\text{sup}} \\
	&\leq \sum_{n=j+1}^i |a_n|\ |\sin(n\pi x)|\ \ \ \ \ \text{by the Triangle Inequality} \\
	&\leq \sum_{n=j+1}^i |a_n|\ \ \ \ \ \text{since $|\sin(n\pi x)| \leq 1$ for any $n$.} \\
	&< \epsilon
\end{align*}
Thus $f_k$ is a Cauchy seqeunce, and since $C([0,1])$ is complete, the limit of $f_k$, which is $f$, must be an element of $C([0,1])$. \hfill $\square$

\subsection*{ b)}
\emph{Prove that the set $A$ defined by}
\begin{equation*}
	A = \{f \in C([0,1])\ |\ f \text{ is of the form (\ref{form_1}) and } \norm{f}_{\text{sup}} \leq 1\}
\end{equation*}
\emph{is not precompact in $(C([0,1]), \norm{\cdot}_{\text{sup}})$.} \\

Consider the family of functions $\mathcal{F} = \{\sin(n\pi x)\}$ for $n = 1, 2, \dots$, and choose any $\delta > 0$.  The choose $N > \frac{2}{\delta}$.  Then the period of $f_N = \sin(N\pi x)$ is $\frac{2}{N} < \delta$.  Then choose a minimum $x_{\text{min}}$ and a maximum $x_{\text{max}}$ of $f_N$ such that $|x_{\text{min}} - x_{\text{max}}| < \delta$ (this can be done since the period of $f_N$ is less than $\delta$).  Since $f_N(x_{\text{min}}) = -1$ and $f_N(x_{\text{max}}) = 1$, then $|f_N(x_{\text{min}}) - f_N(x_{\text{max}})| = 2$.  Thus the family $\mathcal{F}$ is not equicontinuous, and since $\mathcal{F} \subset A$, $A$ is not equicontinuous.  By the Arzel\`a-Ascoli Theorem, $A$ is not precompact. \hfill $\square$

\subsection*{ c)}
\emph{Prove that the set $B$ deinfed by}
\begin{equation*}
	B = \{f \in C([0,1])\ |\ f \text{ is of the form (\ref{form_1}) and } \sum_{n=1}^\infty n^2|a_n|^2 \leq 1\}
\end{equation*}
\emph{is precompact in $(C([0,1]), \norm{\cdot}_{\text{sup}})$.}\\

First we show $B$ is equicontinuous.  Pick $\epsilon >0$ and $x_1 \in [0,1]$.  Assume $d(x_1, x_2) < \qty(\frac{2\epsilon}{\pi})^2$.  Then,
\begin{align*}
	|d(f(x_1), f(x_2))| &= |f(x_1) - f(x_2)| \\
	&= \left|\sum_{n=1}^\infty a_n\qty(\sin(n\pi x_1) - \sin(n\pi x_2))\right| \\
	&= \left|\int_{x_2}^{x_1}\sum_{n=1}^\infty a_n \pi n \cos(n\pi x) \dd x\right| \ \ \ \ \ \text{by the Fundamental Theorem of Calculus} \\
	&= \left|\sum_{n=1}^\infty a_n\pi n \int_{x_2}^{x_1} \cos(n\pi x) \dd x\right| \\
	&= \pi \left|\lim_{N\rightarrow\infty}\qty(\sum_{n=1}^N n a_n \int_{x_2}^{x_1} \cos(n\pi x)\dd x)\right| \\
	&= \pi \left|\lim_{N\rightarrow\infty} \int_{x_2}^{x_1}\sum_{n=1}^N n a_n \cos(n\pi x) \dd x\right| \\
	&\leq \pi\left|\lim_{N\rightarrow\infty} |x_1 - x_2|^{1/2}\qty(\int_{x_2}^{x_1} \qty(\sum_{n=1}^N n a_n \cos(n\pi x))^2 \dd x)^{1/2}\right| \ \ \ \ \ \text{by Problem 2} \\
	&\leq \pi\sqrt{|x_1 - x_2|}\left|\lim_{N\rightarrow\infty} \qty(\int_0^1 \qty(\sum_{n=1}^N n a_n \cos(n\pi x))^2 \dd x)^{1/2}\right|
\end{align*}
We can change the limits of integration since the integrand is positive, and since $|x_1 - x_2| < 1$.  Note that
\begin{align*}
	\int_0^1\qty(\sum_{n=1}^N n a_n \cos(n\pi x))^2 \dd x &= \int_0^1\sum_{n=1}^N n^2 a_n^2 \cos^2(n\pi x) \dd x \\
	&\hphantom{mmmmm}\text{since $\displaystyle\int_0^1 \cos(n\pi x)\cos(m \pi x) \dd x = 0$ for $m \neq n$} \\
	&= \sum_{n=1}^N n^2 a_n^2 \int_0^1 \cos^2(n\pi x) \dd x \\
	&= \sum_{n=1}^N n^2 a_n^2 \qty(\frac{1}{2})\\
	&\hphantom{mmmmm}\text{since $\displaystyle\int_0^1 \cos^2(n\pi x) = \frac{1}{2}$ for every integer $n$} \\
	&= \frac{1}{2} \sum_{n=1}^N n^2 a_n^2
\end{align*}
Thus,
\begin{align*}
	|d(f(x_1), f(x_2))| &\leq \pi\sqrt{|x_1 - x_2|}\left|\lim_{N\rightarrow\infty} \qty(\int_0^1 \qty(\sum_{n=1}^N n a_n \cos(n\pi x))^2 \dd x)^{1/2}\right| \\
	&= \pi\sqrt{|x_1 - x_2|}\left|\lim_{N\rightarrow\infty} \qty(\frac{1}{2}\sum_{n=1}^N n^2 a_n^2)^{1/2}\right| \\
	&= \frac{\pi}{2}\sqrt{|x_1 - x_2|}\left|\lim_{N\rightarrow\infty}\qty(\sum_{n=1}^N n^2 a_n^2)^{1/2}\right| \\
	&= \frac{\pi}{2}\sqrt{|x_1 - x_2|}\left|\qty(\sum_{n=1}^\infty n^2 a_n^2)^{1/2}\right| \\
	&\leq \frac{\pi}{2}\sqrt{|x_1 - x_2|}\qty(\sum_{n=1}^\infty n^2 |a_n|^2)^{1/2}\ \ \ \ \ \text{by the Triangle Inequality} \\
	&\leq \frac{\pi}{2}\sqrt{|x_1 - x_2|}\ \ \ \ \ \text{by the definition of the set $B$} \\
	&< \epsilon\ \ \ \ \ \text{by assumption}
\end{align*}
Thus $B$ is equicontinuous.

Next we show boundedness.  From the above calculation for equicontinuity, take $x_2 = 0$, and let $x_1 \in [0,1]$.  Then
\begin{align*}
	|f(x_1)| = |f(x_1) - 0| &= |f(x_1) - f(x_2)| \\
	&\leq \frac{\pi}{2}\sqrt{|x_1 - x_2|} \\
	&\leq \frac{\pi}{2}\sqrt{|x_1|} \\
	&\leq \frac{\pi}{2}\sqrt{|1|}\ \ \ \ \ \text{since $x_1 \leq 1$}
	&= \frac{\pi}{2}
\end{align*}
Thus each function in $B$ is bounded by $\frac{\pi}{2}$, and so $B$ is bounded.  Since it is also equicontinuous, then by the Arzel\`a-Ascoli Theorem, $B$ is precompact. \hfill $\square$

\end{document}
