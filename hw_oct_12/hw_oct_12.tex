\documentclass[12pt]{article}
\textwidth=17cm \oddsidemargin=-0.9cm \evensidemargin=-0.9cm
\textheight=23.7cm \topmargin=-1.7cm

\usepackage{amssymb, amsmath, amsfonts}
\usepackage{moreverb}
\usepackage{graphicx}
\usepackage{enumerate}
\usepackage{graphics}
\usepackage{color}
\usepackage{array}
\usepackage{float}
\usepackage{hyperref}
\usepackage{textcomp}
\usepackage{alltt}
\usepackage{physics}
\usepackage{mathtools}
\usepackage{tikz}
\usetikzlibrary{positioning}
\usetikzlibrary{arrows}
\usepackage{pgfplots}
\usepackage{bigints}
\allowdisplaybreaks

\newcommand{\suchthat}{\, \mid \,}
\renewcommand{\theenumi}{\alph{enumi}}
\newcommand\Wider[2][3em]{%
\makebox[\linewidth][c]{%
  \begin{minipage}{\dimexpr\textwidth+#1\relax}
  \raggedright#2
  \end{minipage}%
  }%
}

\setcounter{section}{-1}

\title{\bf HW \#2}
\author{\bf Sam Fleischer}
\date{\bf October 12, 2015}

\begin{document}
{\bf MATH 201A \hfill Applied Analysis \ \ \ \ \ \hfill Fall 2015} 

{\let\newpage\relax\maketitle}

\section*{Problem 1}
{\it Let $(X,d)$ be a metric space, with $d(x, y) = 1 - \delta_{x,y}$, for all $x,y \in X$.  Prove that $X$ is compact if and only if $X$ is a finite set.} \\

\noindent ``$\implies$'' \\

Let $X$ be compact.  Then choose $C = \bigcup_{x\in X}B_{\epsilon}(x)$ as an open cover of $X$ for any $0 < \epsilon < 1$.  Since $X$ is compact, there exists a finite subcover of $C$, say $\tilde{C} = \bigcup_{i = 1}^{n}B_{\epsilon}(x_i)$.  However, if $x \in B_{\epsilon}(x_i)$, then $d(x, x_i) < \epsilon < 1$.  This means $d(x, x_i) = 0$, which implies $x = x_i$.  Thus $B_{\epsilon}(x_i) = \{x_i\}$ for $i = 1, \dots n$.  Then $\tilde{C} = \bigcup_{i = 1}^{n}\{x_i\}$.  Since this covers $X$, $X \subset \tilde{C}$.  But $\tilde{C}$ is finite, and so $X$ is finite. \\

\noindent ``$\Longleftarrow$'' \\

Let $X = \{x_1, \dots, x_n\}$ be finite, and let $C = \bigcup_{i\in I}C_i$ be an open cover of $X$.  Then $\forall x_k \in X$, there is at least one $i$ such that $x_k \in C_i$.  Choose one, say $C_k$.  Then $X \subset \bigcup_{k=1}^{n}C_k$, which is a finite open subcover.  Thus $X$ is compact. \hfill $\square$

\section*{Problem 2}
{\it Give an example of a continuous function $f : \mathbb{R} \rightarrow \mathbb{R}$ such that there is a non-empty closed set $F \subset \mathbb{R}$, with $f(F)$ open.} \\

\noindent\textbf{Trivial solution}: Let $f(x) = x$ and let $F = \mathbb{R}$.  Then $F$ is a non-empty closed set since $F^C = \emptyset$ is open.  Then $f(F) = \mathbb{R}$ is open.

\noindent\textbf{Non-trivial solution}: Define $f$ as
\begin{align*}
    f(x) &= \left\{\begin{array}{lcl}
        \dfrac{3}{2^{n+2}}x + \dfrac{2^{n+1} - 3n - 2}{2^{n+1}} &,& x \in [2n, 2n+1], n = 0, 1, \dots \\[.3cm]
        \dfrac{-1}{2^{n+2}}x + \dfrac{2^{n+1} + n}{2^{n+1}} &,& x \in [2n+1, 2(n+1)], n = 0, 1, \dots
    \end{array}\right. \\
    f(-x) &= -f(x)
\end{align*}
and let $F = \bigcup_{n=0}^{\infty}\qty([2n,2n+1]\cup[-(2n+1),-2n])$.  Then $f$ is continuous on $\mathbb{R}$, $F$ is closed (since its complement is a union of open sets), and $f(F) = (-1, 1)$, which is an open set.

\section*{Problem 3}
{\it Let $(X,d)$ be a metric space and $F$ and $K$ two non-empty subsets of $X$.  Assume $F$ is closed and $K$ is compact.  Define}
\begin{equation}
	\label{problem_3_definition}
	d(K, F) = \inf\{d(x, y) | x \in K, y \in F\}.
\end{equation}
{\it Prove that $d(K, F) > 0$ if and only if $K \cap F = \emptyset$.} \\

\noindent ``$\implies$'' \\

Let $d(K, F) > 0$.  Then $\forall k\in K$ and $f\in F$, $d(k, f) > 0$ which implies, $\forall k\in K$ and $f\in F$, $k \neq f$.  Thus $\forall k \in K$, $k\notin F$, and $\forall f \in F$, $f \notin K$.  So $K \cap F = \emptyset$. \\

\noindent ``$\Longleftarrow$'' \\

Let $d(K, F) \not> 0$.  Then, since $d(K,F) \not< 0$, then $d(K,F) = 0$.  Then $\forall \epsilon > 0$, $\exists k \in K, f \in F$ such that $d(k, f) < \epsilon$.  Now construct sequences $(k_n) \in K$ and $(f_n) \in F$ such that $d(k_n, f_n) < \frac{1}{n}$.  Since $K$ is compact, there is a subsequence $(k_{n_l})$ that converges to some limit $\tilde{k} \in K$.  By the triangle inequality,
\begin{align*}
    d(f_{n_l}, \tilde{k}) &\leq d(f_{n_l}, k_{n_l}) + d(k_{n_l}, \tilde{k})
\end{align*}
Since $\lim_{l\rightarrow\infty} d(f_{n_l}, k_{n_l}) = 0$ and $\lim_{l\rightarrow\infty} d(k_{n_l}, \tilde{k}) = 0$, then $\lim_{l\rightarrow\infty} d(f_{n_l}, \tilde{k}) = 0$, which shows the subsequence $(f_{n_l})$ converges to $k$.  However, $f$ is closed, which means every convergent sequence in $F$ converges to a limit in $F$, and since limits are unique, $k \in F$.  Thus $K \cap F \neq \emptyset$. \hfill $\square$

\section*{Problem 4}
{\it Consider the space $X$ of all bounded real-valued functions defined on the interval $[0,1] \subset \mathbb{R}$.  For all $f,g \in X$, define $d(f,g)$ by}
\begin{equation}
	\label{problem_4_definition}
	d(f, g) = \sup\{|f(x) - g(x)| | x \in [0,1]\}.
\end{equation}

\subsection*{ a)}
{\it Prove that $d$ is a metric on $X$.}

\subsection*{ b)}
{\it Prove that the metric space $(X, d)$ is not separable.}

\section*{Problem 5}
{\it Let $(X,d)$ be a metric space and, for each $i = 1, \dots, n$, let $K_i \subset X$ be compact.}

\subsection*{ a)}
{\it Prove that $\bigcap_{i=1}^{n}K_i$ is compact.}

\subsection*{ b)}
{\it Prove that $\bigcup_{i=1}^{n}K_i$ is compact.}

\subsection*{ c)}
{\it Are the union and intersection of an arbitrary family of compact subsets also compact?  Why (not)?}

\section*{Problem 6}
{\it Let $f \in C([0,1])$ be such that $\in_0^1x^nf(x)dx = 0$ for all integers $n \geq 0$.  Prove that $f(x) = 0$, for all $x \in [0,1]$.}

\section*{Problem 7}
{\it Let $(p_n)$ be a sequence of real-valued polynomial functions defined on the interval $[0,1]$ with bounded degree, i.e., there exists $0 \leq D \in \mathbb{Z}$, and sequences of real numbers $(a_n(k))_{n=1}^{\infty}$, $k = 0, \dots, D$, such that}
\begin{equation}
	\label{problem_7_polynomial}
	p_n(x) = a_n(0) + a_n(1)x + \dots + a_n(D)x^D,\ \ x \in [0,1].
\end{equation}

\subsection*{ a)}
{\it Prove that if $\norm{p_n}_\infty \rightarrow 0$, then $\lim_{n\rightarrow\infty}\max_{0\leq k\leq D}a_n(k) = 0$ (Hint: try induction on $D$).}

\subsection*{ b)}
{\it Show that the assumption of a uniform bound on the degree of $p_n$ is essential for the implication in part a) to hold.  Specifically, find a sequence of polynomials $p_n(x) = \sum_{k=0}^{D_n}a_n(k)x^k$, such that $\norm{p_n}_\infty \rightarrow 0$ and}
\begin{equation}
	\label{problem_7b_condition}
	\overline{\lim_n} \max_{0\leq k \leq D_n} |a_n(k)| = 1
\end{equation}

\end{document}
