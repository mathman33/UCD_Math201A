\documentclass[12pt]{article}
\textwidth=17cm \oddsidemargin=-0.9cm \evensidemargin=-0.9cm
\textheight=23.7cm \topmargin=-1.7cm

\usepackage{amssymb, amsmath, amsfonts}
\usepackage{moreverb}
\usepackage{graphicx}
\usepackage{enumerate}
\usepackage{graphics}
\usepackage{color}
\usepackage{array}
\usepackage{float}
\usepackage{hyperref}
\usepackage{textcomp}
\usepackage{alltt}
\usepackage{physics}
\usepackage{mathtools}
\usepackage{tikz}
\usetikzlibrary{positioning}
\usetikzlibrary{arrows}
\usepackage{pgfplots}
\usepackage{bigints}
\allowdisplaybreaks

\newcommand{\suchthat}{\, \mid \,}
\renewcommand{\theenumi}{\alph{enumi}}
\newcommand\Wider[2][3em]{%
\makebox[\linewidth][c]{%
  \begin{minipage}{\dimexpr\textwidth+#1\relax}
  \raggedright#2
  \end{minipage}%
  }%
}

\setcounter{section}{-1}

\title{\bf HW \#1}
\author{\bf Sam Fleischer}
\date{\bf October 5, 2015}

\begin{document}
{\bf MATH 201A \hfill Applied Analysis \ \ \ \ \ \hfill Fall 2015} 

{\let\newpage\relax\maketitle}

\section*{Problem 1}
{\it Let $(X,d)$ be a metric space, and let $x, y, w, z \in X$.}

\subsection*{ a)}
{\it Prove that $d(x, y) \geq |d(x, z) - d(z, y)|$.} \\

By the triangle inequality, $d(x, z) \leq d(x, y) + d(z, y)$ and $d(z, y) \leq d(x, y) + d(x, z)$.  These are equivalent to $d(x, y) \geq d(x, z) - d(z, y)$ and $d(x, y) \geq d(z, y) - d(x, z)$.  Thus, $d(x, y) \geq |d(x, z) - d(z, y)|$. \hfill $\square$

\subsection*{ b)}
{\it Prove that $d(x, y) + d(z, w) \geq |d(x, z) - d(y, w)|$.} \\

By the triangle inequality, $d(x, z) \leq d(x, w) + d(z, w)$ and $d(x, w) \leq d(x, y) + d(y, w)$.  By substitution, $d(x, y) + d(y, w) + d(w, z) \geq d(x, z)$, or
\begin{equation}
	d(x, y) + d(z, w) \geq d(x, z) - d(y, w)
	\label{1.b.1}
\end{equation}
Again by the triangle inequality, $d(y, z) \leq d(x, y) + d(x, z)$ and $d(y, z) \leq d(y, z) + d(w, z)$.  By substitution, $d(x, y) + d(z, w) + d(x, z) \geq d(y, w)$, or
\begin{equation}
	d(x, y) + d(z, w) \geq d(y, w) - d(x, z)
	\label{1.b.2}
\end{equation}
Thus, combining (\ref{1.b.1}) and (\ref{1.b.2}),
\begin{equation}
	d(x, y) + d(z, w) \geq |d(x, z) - d(y, w)|
\end{equation}

\subsection*{ c)}
{\it Let $(x_n)$ and $(y_n)$ be converging sequences in $X$ such that $\lim_{n\rightarrow\infty} x_n = x$ and $\lim_{n\rightarrow\infty} y_n = y$.  Prove that $\lim_{n\rightarrow\infty}d(x_n, y_n) = d(x, y)$.} \\

By the definition of limits, $\forall \frac{\epsilon}{2} > 0$, $\exists N_1, N_2 \in \mathbb{N}$ such that $n > N_1 \implies d(x_n, x) < \frac{\epsilon}{2}$ and $n > N_2 \implies d(y_n, y) < \frac{\epsilon}{2}$.  Then for $n > \max\{N_1, N_2\}$, and by the triangle inequality applied twice, $d(x_n, y_n) \leq d(x_n, x) + d(x, y) + d(y_n, y)$, or
\begin{equation}
	d(x_n, y_n) - d(x, y) \leq d(x_n, x) + d(y_n, y) < \frac{\epsilon}{2} + \frac{\epsilon}{2} = \epsilon
	\label{1.c.1}
\end{equation}
Again, by the triangle inequality applied twice, $d(x, y) \leq d(x_n, x) + d(x_n, y_n) + d(y_n, y)$, or
\begin{equation}
	d(x, y) - d(x_n, y_n) \leq d(x_n, x) + d(y_n, y) < \frac{\epsilon}{2} + \frac{\epsilon}{2} = \epsilon
	\label{1.c.2}
\end{equation}
Thus, combining (\ref{1.c.1}) and (\ref{1.c.2}),
\begin{equation}
	|d(x_n, y_n) - d(x, y)| < \epsilon
\end{equation}
which proves $\lim_{n\rightarrow\infty}d(x_n, y_n) = d(x, y)$. \hfill $\square$

\section*{Problem 2}
{\it Show that the limit of a convergent sequence in a metric space is unique.  I.e., if, for a sequence $(x_n)$ in a metric space $(X, d)$, and $x, y\in X$, $x_n \rightarrow x$ and $x_n \rightarrow y$, then $x = y$.} \\

Assume $x \neq y$.  Then $d(x, y) > 0$.  By the definition of limits, $\forall \epsilon > 0$, $\exists N_1, N_2 \in \mathbb{N}$ such that $n > N_1 \implies d(x_n, x) < \epsilon$ and $n > N_2 \implies d(x_n, y) < \epsilon$.  Now suppose $\epsilon = \frac{1}{2}d(x, y)$.  Then if $n > \max\{N_1, N_2\}$, then $d(x_n, x) < \frac{1}{2}d(x, y)$ and $d(x_n, y) < \frac{1}{2}d(x, y)$.  Adding these inequalities yields
\begin{equation}
	d(x_n, x) + d(x_n, y) < d(x, y)
\end{equation}
which contradicts the triangle inequality.  Thus, $x = y$, i.e. the limit of a convergent sequence in a metric space is unique. \hfill $\square$

\section*{Problem 3}
{\it Let $(a_n)$ be a sequence in $\mathbb{R}$.}

\subsection*{ a)}
{\it Prove that there exists a subsequence $(a_{n_{k}})_{k=1}^{\infty}$ of $(a_n)$ such that $\lim_{k\rightarrow\infty}a_{n_{k}} = \underline{\lim}\ a_n$.} \\

There are three cases, either $\underline{\lim}\ a_n = \infty$, $-\infty$, or $L \in \mathbb{R}$.  It suffices to show we can construct a suitable subsequence in each case.

\subsubsection*{Case 1}

Suppose $\underline{\lim}\ a_n = -\infty$.  First, choose an arbitrary $a_{n_1}$.  Next, choose $a_{n_2}$ such that $a_{n_2} < -2$.  Then choose $a_{n_3}$ such that $a_{n_3} < -3$, and so on such that $a_{n_k} < -k$.  Then $(a_{n_k})$ diverges to $-\infty$, i.e. $\lim_{k\rightarrow\infty}a_{n_{k}} = -\infty = \underline{\lim}\ a_n$.

\subsubsection*{Case 2}

Suppose $\underline{\lim}\ a_n = \infty$.  First, choose an arbitrary $a_{n_1}$.  Next, choose $a_{n_2}$ such that $a_{n_2} \geq a_{n_1}$.  Then choose $a_{n_3}$ such that $a_{n_3} \geq a_{n_2}$, and so on such that $a_{n_1} \leq a_{n_2} \leq \dots \leq a_{n_k} \leq \dots$.  This sequence $(a_{n_k})$ does not have a real limit, since that would contradict $\underline{\lim}\ a_n = \infty$.  Thus $(a_{n_k})$ diverges to $\infty$, i.e. $\lim_{k\rightarrow\infty}a_{n_{k}} = \infty = \underline{\lim}\ a_n$.

\subsubsection*{Case 3}

Suppose $\underline{\lim}\ a_n = L \in \mathbb{R}$.  First, choose an arbitrary $a_{n_1}$.  Next, choose $a_{n_2}$ such that $|L - a_{n_2}| < \frac{1}{2}$.  Then choose $a_{n_3}$ such that $|L - a_{n_3}| < \frac{1}{3}$ and so on such that $|L - a_{n_k}| < \frac{1}{k}$.  Then by the Archimedian principle, $\forall\epsilon > 0$, $\exists N \in \mathbb{N}$ such that $k > N \implies |L - a_{n_k}| < \epsilon$, i.e. $\lim_{k\rightarrow\infty} a_{n_k} = L = \underline{\lim}\ a_n$.

\hfill $\square$

\subsection*{ b)}
{\it Prove that $(a_n)$ converges to $a \in \mathbb{R}$ if and only if $\underline{\lim}\ a_n = \overline{\lim}\ a_n = a$.}

\subsubsection*{``$\implies$''}

Let $(a_n) \rightarrow a$.  Then $\forall\epsilon > 0$, $\exists N \in\mathbb{N}$ such that $n \geq N \implies |a_n - a| < \epsilon$.  Then
\begin{align*}
	a - \epsilon < \inf\{a_k | k \geq n\} &< a + \epsilon \\
	\iff -\epsilon < \inf\{a_k | k \geq n\} - a &< \epsilon \\
	\iff |\inf\{a_k | k \geq n\} - a| &< \epsilon \\
	\iff \lim_{n\rightarrow\infty}\left(\inf\{a_k | k \geq n\}\right) &= a \\
	\iff \underline{\lim}\ a_n &= a
\end{align*}
Similarly,
\begin{align*}
	a - \epsilon < \sup\{a_k | k \geq n\} &< a + \epsilon \\
	\iff -\epsilon < \sup\{a_k | k \geq n\} - a &< \epsilon \\
	\iff |\sup\{a_k | k \geq n\} - a| &< \epsilon \\
	\iff \lim_{n\rightarrow\infty}\left(\sup\{a_k | k \geq n\}\right) &= a \\
	\iff \overline{\lim}\ a_n &= a
\end{align*}
Thus, $\underline{\lim}\ a_n = \overline{\lim}\ a_n = a$

\subsubsection*{``$\Longleftarrow$''}

Let $\underline{\lim}\ a_n = \overline{\lim}\ a_n = a$.  Then $\forall\epsilon > 0$, $\exists N_1, N_2$ such that $n \geq N_1 \implies |\inf\{a_k | k \geq n\} - a| < \epsilon$ and $n \geq N_2 \implies |\sup\{a_k | k \geq n\} - a| < \epsilon$.  Let $\epsilon > 0$ and $n \geq \max\{N_1, N_2\}$.  For ease, define $K \equiv \{a_k | k \geq n\}$  Since the infimum of a set is always less than or equal to the supremum of that set, we can write
\begin{align*}
	-\epsilon < \inf K - a &\leq \sup K - a < \epsilon \\
	\iff -\epsilon + a < \inf K &\leq \sup K < \epsilon + a
\end{align*}
By the definition of infimum and supremum,
\begin{align*}
	-\epsilon + a < \inf K \leq\ &a_n \leq \sup K < \epsilon + a \\
	\implies \epsilon + a <\ &a_n < \epsilon + a \\
	\iff |a_n - a| < \epsilon
\end{align*}
Thus, $(a_n) \rightarrow a$. \\

\noindent Thus, $(a_n) \rightarrow a \iff \underline{\lim}\ a_n = \overline{\lim}\ a_n = a$. \hfill $\square$

\section*{Problem 4}
{\it Let $(X,d)$ be a metric space.  Prove the statements in Proposition 1.37 in the textbook:}

\subsection*{ a)}
{\it The empty set $\emptyset$ and $X$ itself are both open and closed sets in $(X, d)$.} \\

It is vacuously true that the empty set $\emptyset$ is open.  Thus $X$ is closed.  Let $x \in X$ and choose $r \in \mathbb{R}$.  Then $y\in B_r(x) \implies y \in X$ since $B_r(x) \subseteq X$.  Thus $X$ is open and the empty set $\emptyset$ is closed. \hfill $\square$

\subsection*{ b)}
{\it The intersection of a finite collection of open sets is open.} \\

Let $A = \bigcap\limits_{k=1}^{n} A_k$ be the intersection of a finite collection of open sets, and let $x \in A$.  Then $\exists r_1, \dots r_n$ such that $y \in B_{r_i}(x) \implies y \in A_i$ for $i = 1, \dots, n$.  Then let $r = \min\{r_1, \dots r_n\}$.  Then $y \in B_r(x) \implies y \in B_{r_i}(x)$ for $i = 1, \dots n$.  Thus $y \in A$.  Thus the intersection of a finite collection of open sets is open. \hfill $\square$

\subsection*{ c)}
{\it The union of an arbitrary collection of open sets is open.} \\

Let $A = \bigcup\limits_{i\in I} A_i$ be the union of an arbitrary collection of open sets, and let $x \in A$.  Then $x \in A_k$ for some $k \in I$.  Since $A_k$ is open, $\exists r$ such that $y \in B_r(x) \implies y \in A_k$.  But since $A_k \subseteq A$, $y \in A$.  Thus the union of an arbitrary collection of open sets is open. \hfill $\square$

\subsection*{ d)}
{\it The union of a finite collection of closed sets is closed.} \\

Let $A = \bigcup\limits_{k=1}^{n} A_k$ be the union of a finite collection of closed sets.  By De Morgan's Law in Set Theory, $A^C = \bigcap\limits_{k=1}^{n} A_k^C$.  $A^C$ is open since each $A_k^C$ is open and the intersection of a finite collection of open sets is open.  Since $A_C$ is open, $A$ is closed.  Thus the union of a finite collection of closed sets is closed. \hfill $\square$

\subsection*{ e)}
{\it The intersection of an arbitrary collection of closed sets is closed.} \\

Let $A = \bigcap\limits_{i\in I} A_i$ be the intersection of an arbitrary collection of closed sets.  By De Morgan's Law in Set Theory, $A^C = \bigcup\limits_{i\in I} A_i^C$.  $A^C$ is open since each $A_i^C$ is open and the union of an arbitrary collection of open sets is open.  Since $A^C$ is open, $A$ is closed.  Thus the intersection of an arbitrary collection of closed sets is closed. \hfill $\square$

\section*{Problem 5}
{\it Let $(X, d_X)$ and $(Y, d_Y)$ be metric spaces, $f: X \rightarrow Y$ a ontinuous function, and $B \subset Y$ a closed set.  Prove that $A$ defined by}
\begin{align*}
	A = \{x \in X | f(x) \in B\}
\end{align*}
{\it is a closed set.} \\

Let $a \in A^C$.  Then $f(a) \notin B$.  Then $f(a) \in B^C$.  Since $B$ is closed, $B^C$ is open, and thus $\exists \epsilon$ such that $y \in B_\epsilon(f(a)) \implies y \in B^C$.  By the definition of continuous functions, $\exists \delta$ such that $x \in B_\delta(a) \implies f(x) \in B_\epsilon(f(a))$, which then implies $f(x) \in B^C$.  Thus, $f(x) \notin B \implies x \notin A \implies x \in A^C$.  Thus $A^C$ is open, which implies $A$ is closed. \hfill $\square$

\section*{Problem 6}
{\it Let X be a Banach space and let $(x_n)$ be a sequence in $X$ such that $\sum_{n=1}^\infty \norm{x_n} = 1$.}

\subsection*{ a)}
{\it Prove that the series $\sum_{n=1}^\infty x_n$ converges to a limit $x \in X$.} \\

Convergence of a series is equivalent to convergence of the sequence of partial sums, so since $\sum_{n=1}^\infty \norm{x_n} = 1$, then $\lim_{k\rightarrow\infty}\sum_{n=1}^k \norm{x_n} = 1$.  The sequence is Cauchy since the real numbers are complete, so $\forall\epsilon>0$, $\exists N$ such that $a \geq b \geq N \implies \left|\sum_{n=1}^a \norm{x_n} - \sum_{n=1}^b \norm{x_n}\right| < \epsilon$, or $\left|\sum_{n=b+1}^a \norm{x_n}\right| < \epsilon$.  Since the norm is always positive, the sum of norms is positive.  Thus, it can also be written without the absolute value: $\sum_{n=b+1}^a \norm{x_n} < \epsilon$.

Now let $\epsilon > 0$ and $a \geq b \geq N$.  Then
\begin{align*}
	\norm{\sum_{n=1}^{a} x_n - \sum_{n-1}^{b} x_n} &= \norm{\sum_{n=b+1}^a x_n} \\
	&\leq \sum_{n=b+1}^a \norm{x_n}\ \ \text{(by the triangle inequality of normed metric spaces)} \\
	&< \epsilon
\end{align*}
Thus $\qty(\sum_{n=1}^k x_n)_k$ is a Cauchy sequence in $X$, and since all Banach spaces are complete, $\qty(\sum_{n=1}^k x_n)_k$ converges to a limit $x \in X$.  Any sequence of partial sums can be written as a series, and so $\sum_{n=1}^\infty x_n$ converges to a limit $x \in X$. \hfill $\square$

\subsection*{ b)}
{\it Prove that for any subsequence $(x_{n_k})_{k=1}^\infty$ of $(x_n)$, the series $\sum_{k=1}^\infty x_{n_k}$ also converges and that the norm of its limit is bounded by $1$.} \\

By part a), $\sum_{n=1}^\infty x_n$ is convergent, and therefore Cauchy.  Thus $\forall\epsilon>0$, $\exists N$ such that $a \geq b \geq N \implies \left|\sum_{n=b+1}^{a} x_n\right| < \epsilon$.  Then let $\epsilon>0$ and pick $c,d\in\mathbb{N}$ such that $n_c \geq n_d \geq N$.  Then $\left|\sum_{k=d+1}^{c} x_{n_k}\right| < \epsilon$.  Thus, the subsequence $(x_{n_k})_{k=1}^\infty$ is a Cauchy sequence in $X$, and since all Banach spaces are complete, $(x_{n_k})_{k=1}^\infty$ is convergent.

Now consider the norm of the limit of partial sums of the subsequence: $\norm{\lim_{l\rightarrow\infty}\sum_{k=1}^{l} x_{n_k}}$.  By the triangle inequality of normed metric spaces,
\begin{align*}
	\norm{\lim_{l\rightarrow\infty}\sum_{k=1}^{l} x_{n_k}} \leq \lim_{l\rightarrow\infty}\sum_{k=1}^{l}\norm{x_{n_k}}
\end{align*}
Since all norms are positive
\begin{align*}
	\lim_{l\rightarrow\infty}\sum_{k=1}^{l}\norm{x_{n_k}} &\leq \lim_{k\rightarrow\infty}\sum_{n=1}^k \norm{x_{n_k}} \\
	&= \sum_{n=1}^\infty \norm{x_{n_k}} = 1
\end{align*}
Thus, the norm of the limit of the subseries is bounded by $1$. \hfill $\square$

\end{document}
