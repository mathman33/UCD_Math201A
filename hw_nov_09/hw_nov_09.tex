\documentclass[12pt]{article}
\textwidth=17cm \oddsidemargin=-0.9cm \evensidemargin=-0.9cm
\textheight=23.7cm \topmargin=-1.7cm

\usepackage{amssymb, amsmath, amsfonts}
\usepackage{moreverb}
\usepackage{graphicx}
\usepackage{enumerate}
\usepackage{graphics}
\usepackage{color}
\usepackage{array}
\usepackage{float}
\usepackage{hyperref}
\usepackage{textcomp}
\usepackage{alltt}
\usepackage{physics}
\usepackage{mathtools}
\usepackage{amsthm}
\usepackage{tikz}
\usetikzlibrary{positioning}
\usetikzlibrary{arrows}
\usepackage{pgfplots}
\usepackage{bigints}
\allowdisplaybreaks

\theoremstyle{plain}
\newtheorem*{theorem*}{Theorem}
\newtheorem{theorem}{Theorem}
\newtheorem*{lemma*}{Lemma}
\newtheorem{lemma}{Lemma}

% \newenvironment{proof}[1][Proof]{\begin{trivlist}
% \item[\hskip \labelsep {\bfseries #1}]}{\end{trivlist}}
% \newenvironment{definition}[1][Definition]{\begin{trivlist}
% \item[\hskip \labelsep {\bfseries #1}]}{\end{trivlist}}
% \newenvironment{example}[1][Example]{\begin{trivlist}
% \item[\hskip \labelsep {\bfseries #1}]}{\end{trivlist}}
% \newenvironment{remark}[1][Remark]{\begin{trivlist}
% \item[\hskip \labelsep {\bfseries #1}]}{\end{trivlist}}

% \newcommand{\qed}{\nobreak \ifvmode \relax \else
%       \ifdim\lastskip<1.5em \hskip-\lastskip
%       \hskip1.5em plus0em minus0.5em \fi \nobreak
%       \vrule height0.75em width0.5em depth0.25em\fi}

\newcommand{\suchthat}{\, \mid \,}
\renewcommand{\theenumi}{\alph{enumi}}
\newcommand\Wider[2][3em]{%
\makebox[\linewidth][c]{%
  \begin{minipage}{\dimexpr\textwidth+#1\relax}
  \raggedright#2
  \end{minipage}%
  }%
}

\setcounter{section}{-1}

\title{\bf HW \#5}
\author{\bf Sam Fleischer}
\date{\bf November 9, 2015}

\begin{document}
{\bf MATH 201A \hfill Applied Analysis \ \ \ \ \ \hfill Fall 2015} 

{\let\newpage\relax\maketitle}

\section*{Problem 1}
\emph{Let $(X, \mathcal{T})$ be a Hausdorff space and $F,K \subset X$ such that $F$ is closed and $K$ is compact.}

\subsection*{ a)}
\emph{Prove that $K$ is closed.} \\

Pick $y$ in $K^C$.  Then for every $x \in K$, choose an open neighborhood of $x$, $U_x$, and an open neighborhood of $y$, $V_x$, such that $U_x \cap V_x = \emptyset$ for each $x$.  This is possible since $X$ is a Hausdorff space.  Clearly, $\{U_x\}_{x\in K}$ is an open cover of $K$.  Since $K$ is compact, $\exists x_1, \dots, x_n$ such that $\{U_{x_i}\}_{i = 1}^n$ is an open cover of $K$.  Let $V = \bigcap_{i = 1}^n V_{x_i}$.  Then $V$ is open since it is a finite intersection of open neighborhoods.  Let $v \in V$.  Then for $i = 1, \dots, n$, $v \not\in U_{x_i}$.  Then $v \not\in K$, i.e.~$v \in K^C$.  Thus $V \subset K^C$.  Thus $K^C$ contains a neighborhood of each element of $K^C$, and so $K^C \in \mathcal{T}$.  Thus $K$ is closed. \hfill $\square$

\subsection*{ b)}
\emph{Prove that $F \cap K$ is compact.} \\

Choose an open cover $\{G_\alpha\}_\alpha$ of $F \cap K$.  Since $K$ is compact, it is closed (by part a), and since $F$ is also closed, $F \cap K$ is closed, i.e.~$(F \cap K)^C$ is open.  Then $\left\{\{G_\alpha\}_\alpha, (F \cap K)^C\right\}$ is an open cover of $K$.  Then since $K$ is compact, there is a finite open subcover, namely $\left\{\{G_{\alpha_i}\}_{i = 1}^n, (F \cap K)^C\right\}$.  But since $(F \cap K)^C \cap (F \cap K) = \emptyset$, then $\{G_{\alpha_i}\}_{i = 1}^n$ is an open cover of $F \cap K$.  Since this is a subcover of $\{G_\alpha\}$, then $F \cap K$ is compact. \hfill $\square$

\section*{Problem 2}
\emph{Let $(X, \mathcal{T})$ be a topological space and $K_1$, $K_2$ two compact subsets of $X$.}

\subsection*{ a)}
\emph{Prove that $K_1 \cup K_2$ is compact.} \\

Let $\{G_\alpha\}_\alpha$ be an open cover of $K_1 \cup K_2$.  Then $\{G_\alpha\}_\alpha$ is an open cover of both $K_1$ and $K_2$.  Then there are finite subcovers $\{G_{\alpha_i}\}_{i = 1}^n$ and $\{G_{\alpha_j}\}_{j = 1}^m$ of $K_1$ and $K_2$, respectively.  Then $\left\{\{G_{\alpha_i}\}_{i = 1}^n, \{G_{\alpha_j}\}_{j = 1}^m\right\}$ is a finite cover of $K_1 \cup K_2$, and is a subcover of $\{G_{\alpha}\}_\alpha$.  Thus every open cover has a finite subcover, proving $K_1 \cup K_2$ is compact. \hfill $\square$

\subsection*{ b)}
\emph{Assuming $(X, \mathcal{T})$ is Hausdorff, proce that $K_1 \cap K_2$ is compact.} \\

By part 1.a), the compactness of $K_1$ implies its closure.  Thus by part 1.b), $K_1 \cap K_2$ is compact. \hfill $\square$

\section*{Problem 3}
\emph{If $A$ is a subset of a toplogical space, then the \emph{interior} $A^\circ$ of $A$ is the union of all open sets contained in $A$, the \emph{closure} $\overline{A}$ of $A$ is the intersection of all closed sets that contain $A$, and the \emph{boundary} $\partial A$ of $A$ is defined by $\partial A = \overline{A} \cap \overline{A^C}$.}

\begin{lemma}
    \label{awesome_lemma}
    $\overline{A^C} = \qty(A^\circ)^C$
\end{lemma}
\begin{proof}
    Let $\{C_\alpha\}$ be the set of all closed sets containing $A^C$.  Then by the definition of closure, $\overline{A^C} = \bigcap_\alpha C_\alpha$.  Since $A^C \subset C_\alpha$ for all $\alpha$, then $C_\alpha^C \subset A$ for all $\alpha$.  Also, since $C_\alpha$ is closed for all $\alpha$, $C_\alpha^C$ is open for all $\alpha$.  In addition, if $G$ is an open set contained in $A$, then $G = C_\alpha^C$ for some $C_\alpha$.  Then by the definition of interior, $A^\circ = \bigcup_\alpha C_\alpha^C$.  Thus,
    \begin{align*}
        \qty(A^\circ)^C = \qty(\cup_\alpha C_\alpha^C)^C = \cap_\alpha \qty(C_\alpha^C)^C = \cap_\alpha C_\alpha = \overline{A^C}
    \end{align*}
\end{proof}

\begin{lemma}
    \label{second_awesome_lemma}
    $\overline{A}^C = \qty(A^C)^\circ$
\end{lemma}
\begin{proof}
    Let $B = A^C$.  Then by Lemma 1, $\overline{B^C} = \qty(B^\circ)^C$.  Then $\overline{\qty(A^C)^C} = \qty(\qty(A^C)^\circ)^C$.  Thus $\overline{A} = \qty(\qty(A^C)^\circ)^C$.  Thus $\overline{A}^C = \qty(A^C)^\circ$
\end{proof}

\subsection*{ a)}
\emph{Show that a set is closed if and only if it contains its boundary.} \\

\noindent``$\Longrightarrow$''  Let $A$ be closed.  Then $A = \overline{A}$.  Then $\partial A = \overline{A} \cap \overline{A^C} \subset \overline{A} = A$.  Then $A$ contains its boundary. \\

\noindent``$\Longleftarrow$''  Let $A$ contain its boundary, i.e. $\partial A = \overline{A} \cap \overline{A^C} \subset A$.  We want to show $A^C$ is open, i.e.~$A^C = \qty(A^C)^\circ$.  Obviously, $\qty(A^C)^\circ \subset A^C$.  Let $x \in A^C$.  Then $x \not\in A$.  Since $\partial A \subset A$, $x \not\in \partial A$.  Then either $x \not\in \overline{A}$ or $x \not\in \overline{A^C}$, i.e.~either $x \in \overline{A}^C$ or $x \in \overline{A^C}^C$.  By Lemmas \ref{awesome_lemma} and \ref{second_awesome_lemma}, either $x \in \qty(A^C)^\circ$ or $x \in A^\circ$.  But since $x \not\in A$, $x \not\in A^\circ$.  Thus $x \in \qty(A^C)^\circ$.  Then $A^C = \qty(A^C)^\circ$.  Thus $A^C$ is open, proving $A$ is closed. \hfill $\square$

\subsection*{ b)}
\emph{Show that a set is open if any only if it is disjoint from its boundary.}\\

\noindent``$\Longrightarrow$''  Let $A$ be open.  Then $A = A^\circ$.  Then $A \cap \partial A = A \cap \qty(\overline{A} \cap \overline{A^C}) = A \cap \qty(\overline{A} \cap \qty(A^\circ)^C)$ (by Lemma \ref{awesome_lemma}) and thus $A \cap \partial A = A \cap \qty(\overline{A} \cap A^C) = \qty(A \cap A^C) \cap \overline{A} = \emptyset \cap \overline{A} = \emptyset$.  Thus $A$ is disjoint from its boundary. \\

\noindent``$\Longleftarrow$''  Let $A \cap \partial A = \emptyset$, and choose $x \in A$.  Then $x \not\in \partial A$.  Thus $x \not\in \overline{A}$ or $x \not\in \overline{A^C}$.  Since $x \in A$, $x \in \overline{A}$.  Thus $x \not\in \overline{A^C}$.  By Lemma \ref{awesome_lemma}, $x \not\in \qty(A^\circ)^C$.  Thus $x \in A^\circ$.  Since $A^\circ$ is open, there is a neighborhood $G$ of $x$ such that $G \subset A^\circ$.  But $A^\circ \subset A$.  Thus $A$ is open. \hfill $\square$

\subsection*{ c)}
\emph{What are the closure, interior, and boundary of the Cantor set, considered as a subset of $\mathbb{R}$ with its usual topology?  The Cantor set is defined in Example 1.40 of the textbook.}\\

Define the function $f$ whose domain is closed intervals of $\mathbb{R}$ by 
\begin{align*}
    f([a,b]) = \left\{\qty[a, a + \frac{b-a}{3}], \qty[b - \frac{b-a}{3}, b]\right\}
\end{align*}
Define $G_n$ as follows:
\begin{align*}
    G_0 &= \left\{\qty[0,1]\right\} \\
    G_1 &= \left\{\qty[0, \frac{1}{3}], \qty[\frac{2}{3}, 1]\right\} \\
    &\vdots \\
    G_n &= \bigcup\limits_{[a,b] \in G_{n-1}} f([a,b])\\
    &\vdots
\end{align*}
and define $F_n \equiv \bigcup\limits_{[a,b] \in G_n} [a,b]$.  Finally, define the Cantor set $\mathcal{C} = \bigcap\limits_{n=0}^\infty F_n$.  Since for each $n$, $\left|G_n\right| = 2^n$, label each element of $G_n$ as $G_{n,k}$ for $k = 1, \dots, 2^n$.  Note that for each $G_{n,k}$, $\sup\left\{|x_1 - x_2|\ |\ x_1,x_2 \in G_{n,k}\right\} = 3^{-n}$. Next we will show $\mathcal{C}^\circ = \emptyset$, which will show $\mathcal{C} = \overline{\mathcal{C}}$ and $\partial\mathcal{C} = \mathcal{C}$.

Let $x \in \mathcal{C}^\circ$.  Then since $\mathcal{C}^\circ$ is open, there is some open neighborhood $U$ such that $x \in U \subset \mathcal{C}^\circ$.  Since $U$ is an open neighborhood, $\exists \epsilon > 0$ such that $x \in B_\epsilon(x) \subset U \subset \mathcal{C}^\circ$.  Since $\mathcal{C}^\circ \subset \mathcal{C} = \bigcap\limits_{n=0}^\infty F_n$, then $\forall n$, $\exists k$ such that $B_\epsilon(x) \subset G_{n,k}$.  Thus $\forall n$, $\sup\left\{|y_1 - y_2|\ |\ y_1,y_2 \in B_\epsilon(x)\right\} = 2\epsilon < 3^{-n}$, which is a contradiction.  Thus $\mathcal{C}^\circ = \emptyset$, and $\overline{\mathcal{C}} = \mathcal{C}$.  Finally $\partial\mathcal{C} = \overline{\mathcal{C}} \cap \overline{\mathcal{C}^C} = \mathcal{C} \cap \qty(\mathcal{C}^\circ)^C = \mathcal{C} \cap \mathbb{R} = \mathcal{C}$. \hfill $\square$

\section*{Problem 4}
\emph{A topological space is \emph{connected} if it is not the union of two disjoint non-empty open sets.  A subset $Y$ of a topological space $(X, \mathcal{T})$ is called connected if $Y$ is a connected topological space with respect to the relative topology.}

\subsection*{ a)}
\emph{Describe the connected subsets of $(\mathbb{R}, |\cdot|)$.}

\begin{lemma}
    \label{connected_subsets_of_R}
    The connected subsets of $\mathbb{R}$ are intervals.
\end{lemma}
\begin{proof}
    ``$\Longrightarrow$'' Suppose $G \subset \mathbb{R}$ is not an interval.  Then $\exists x,\epsilon$ such that $x \not\in G$ but $x - \epsilon \in G$ and $x + \epsilon \in G$.  Then pick $U_1 = (-\infty, x) \cap G$ and $U_2 = (x, \infty) \cap G$.  Then $U_1$ and $U_2$ are open in the relative topology on $G$ and $U_1 \cap U_2 = \emptyset$.  Thus $G$ is not connected.\\

    \noindent``$\Longleftarrow$'' Suppose $G \subset \mathbb{R}$ is not connected.  Then $G = U\cup V$ where $U,V\in \mathcal{T}$ and $U\cap V = \emptyset$.  $U \in \mathcal{T} \implies U = \cup_{\alpha \in I} (a_\alpha, b_\alpha) \cap G$, where $a_\alpha \neq b_\alpha$ for every $\alpha$ in the index set $I$.  Similarly, $V \in \mathcal{T} \implies V = \cup_{\beta\in J}(c_\beta, d_\beta) \cap G$ where $c_\beta \neq d_\beta$ for every $\beta$ in the index set $J$.  Let $\epsilon = \inf\{|u - v|\ |\ u \in U, v\in V\}$.\\

    \noindent If $\epsilon = 0$, then pick $(u_n)_n \in U$ and $(v_n)_n \in V$ such that $\lim_{n\rightarrow\infty}u_n = \lim_{n\rightarrow\infty}v_n = L$.  If $L \in U$, then $\exists \tilde\epsilon > 0$ such that $B_{\tilde\epsilon}(L) \subset U$.  But since $\lim_{n\rightarrow\infty}v_n = L$, then $\exists N$ such that $n \geq N \implies v_n \in B_{\tilde\epsilon}(L)$, which is a contradiction since $U \cap V = \emptyset$.  Thus $L \not\in U$.  Similarly, $L \not\in V$.  Thus $L \not\in G$.  However, $\exists \overline{\epsilon}$ such that $L \pm \overline{\epsilon} \in U \subset G$ and $L \mp \overline{\epsilon} \in V \subset G$.  Thus $G$ is not an interval. \\

    \noindent If $\epsilon > 0$, then pick $(u_n)_n \in U$ and $(v_n)_n \in V$ such that $\lim_{n\rightarrow\infty}u_n = \lim_{n\rightarrow\infty}v_n \pm \epsilon$.  Then let $L = \lim_{n\rightarrow\infty}u_n \pm \epsilon/2$.  Then $L \not\in U$ and $L \not\in V$ (thus $L \not\in G$) but $\exists \overline{\epsilon}$ such that $L \pm \overline{\epsilon} \in U \subset G$ and $L \mp \overline{\epsilon} \in V \subset G$.  Thus $G$ is not an interval.
\end{proof}

\subsection*{ b)}
\emph{Show that $(\mathbb{R}, |\cdot|)$ is homeomorphic to the open interval $(0,1) \subset \mathbb{R}$ with the relative topology.} \\

Define the function $f$ as
\begin{align*}
    f(x) = \frac{\tan^{-1}(x) + \frac{\pi}{2}}{\pi}
\end{align*}
Then since $\tan^{-1}(x)$ is a continuous bijection from $\mathbb{R}$ to $(-\frac{\pi}{2}, \frac{\pi}{2})$, then since $f(x)$ is a translation of $\tan^{-1}(x)$, then $f(x)$ is a continuous bijection from $\mathbb{R}$ to $(0,1)$.  In addition,
\begin{align*}
    f^{-1}(x) = \tan\qty(\pi x - \frac{\pi}{2})
\end{align*}
is a continuous bijection from $(0,1)$ to $\mathbb{R}$.  Thus $(\mathbb{R}, |\cdot|)$ is homeomorphic to $(0,1)$.

\subsection*{ c)}
\emph{Show that $(\mathbb{R}, |\cdot|)$ is not homeomorphic to $(\mathbb{R}^2, \norm{\cdot})$, where $\norm{\cdot}$ is the Euclidean norm.} \\



\end{document}
