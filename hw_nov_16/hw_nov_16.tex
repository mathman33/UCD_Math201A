\documentclass[12pt]{article}
\textwidth=17cm \oddsidemargin=-0.9cm \evensidemargin=-0.9cm
\textheight=23.7cm \topmargin=-1.7cm

\usepackage{amssymb, amsmath, amsfonts}
\usepackage{moreverb}
\usepackage{graphicx}
\usepackage{enumerate}
\usepackage{graphics}
\usepackage{color}
\usepackage{array}
\usepackage{float}
\usepackage{hyperref}
\usepackage{textcomp}
\usepackage{alltt}
\usepackage{physics}
\usepackage{mathtools}
\usepackage{amsthm}
\usepackage{tikz}
\usetikzlibrary{positioning}
\usetikzlibrary{arrows}
\usepackage{pgfplots}
\usepackage{bigints}
\allowdisplaybreaks
\pgfplotsset{compat=1.12}

\theoremstyle{plain}
\newtheorem*{theorem*}{Theorem}
\newtheorem{theorem}{Theorem}
\newtheorem*{lemma*}{Lemma}
\newtheorem{lemma}{Lemma}

% \newenvironment{proof}[1][Proof]{\begin{trivlist}
% \item[\hskip \labelsep {\bfseries #1}]}{\end{trivlist}}
% \newenvironment{definition}[1][Definition]{\begin{trivlist}
% \item[\hskip \labelsep {\bfseries #1}]}{\end{trivlist}}
% \newenvironment{example}[1][Example]{\begin{trivlist}
% \item[\hskip \labelsep {\bfseries #1}]}{\end{trivlist}}
% \newenvironment{remark}[1][Remark]{\begin{trivlist}
% \item[\hskip \labelsep {\bfseries #1}]}{\end{trivlist}}

% \newcommand{\qed}{\nobreak \ifvmode \relax \else
%       \ifdim\lastskip<1.5em \hskip-\lastskip
%       \hskip1.5em plus0em minus0.5em \fi \nobreak
%       \vrule height0.75em width0.5em depth0.25em\fi}

\newcommand{\suchthat}{\, \mid \,}
\renewcommand{\theenumi}{\alph{enumi}}
\newcommand\Wider[2][3em]{%
\makebox[\linewidth][c]{%
  \begin{minipage}{\dimexpr\textwidth+#1\relax}
  \raggedright#2
  \end{minipage}%
  }%
}

\setcounter{section}{-1}

\title{\bf HW \#6}
\author{\bf Sam Fleischer}
\date{\bf November 16, 2015}

\begin{document}
{\bf MATH 201A \hfill Applied Analysis \ \ \ \ \ \hfill Fall 2015} 

{\let\newpage\relax\maketitle}

\section*{Problem 1}
\emph{Let $k : [0,1]\times[0,1]\rightarrow \mathbb{R}$ be a continuous function.  Define the map $T:C([0,1])\rightarrow C([0,1])$ by}
\begin{align*}
    (Tf)(x) = \int_0^1 k(x,y)f(y)\dd y,\ \ \text{for all } f \in C([0,1])
\end{align*}

\subsection*{ a)}
\emph{Let $\norm{T}$ denote the operator norm of $T$.  Prove}
\begin{equation}
    \label{eq_2}
    \norm{T} = \sup\limits_{x\in[0,1]}\int_0^1|k(x,y)|\dd y
\end{equation}

First, we show $\norm{T} \leq \sup\limits_{x\in[0,1]}\int_0^1|k(x,y)|\dd y$.
\begin{align*}
    \norm{T} = \sup\limits_{\norm{f}=1}\norm{Tf} &= \sup\limits_{\norm{f}=1}\norm{\int_0^1 k(\cdot,y)f(y)\dd y} \\
    &= \sup\limits_{x\in[0,1]}\qty(\sup\limits_{\norm{f}=1}\left|\int_0^1 k(x,y) f(y) \dd y\right|) \\
    &\leq \sup_{x\in[0,1]}\qty(\sup_{\norm{f}=1}\int_0^1|k(x,y)||f(y)|\dd y) \\
    &\leq \sup_{x\in[0,1]}\qty(\sup_{\norm{f}=1}\int_0^1|k(x,y)|\norm{f}\dd y) \\
    &= \sup_{x\in[0,1]}\int_0^1|k(x,y)|\dd y
\end{align*}
Note that since $\int_0^1|k(x,y)|\dd y$ is a continuous functions of $x$, then $\exists x^* \in [0,1]$ such that
\begin{align*}
    \sup\limits_{x\in[0,1]}\int_0^1|k(x,y)|\dd y = \int_0^1|k(x^*,y)|\dd y
\end{align*}
Thus,
\begin{align*}
    \norm{T} \leq \int_0^1|k(x^*,y)|\dd y
\end{align*}
Next, we show $\norm{T} \geq \int_0^1|k(x^*,y)|\dd y$.  Define a sequence of functions $(f_n)_n$
\begin{align*}
    f_n(y) = \frac{k(x^*,y)}{\frac{1}{n} + |k(x^*,y)|}
\end{align*}
Note $f_n(y) \rightarrow \text{sign}(k(x^*,y))$ for each $y \in [0,1]$, i.e.~$f_n \rightarrow \text{sign}(k(x^*,\cdot))$ pointwise.
\begin{align*}
    \sup_{x\in[0,1]}\int_0^1 |k(x,y)| \dd y &= \int_0^1 |k(x^*, y)| \dd y \\
    &= \int_0^1 k(x^*,y)\text{sign}(k(x^*,y)) \dd y \\
    &= \int_0^1 k(x^*,y)\lim\limits_{n\rightarrow\infty}f_n(y) \dd y
\end{align*}
Since $\norm{f_n}_\infty \leq 1$ for all $n$, we can employ Lebesgue's Dominated Convergence Theorem to pull the limit out of the integral:
\begin{align*}
    \int_0^1 k(x^*,y)\lim\limits_{n\rightarrow\infty}f_n(y) \dd y = \lim\limits_{n\rightarrow\infty} \int_0^1 k(x^*,y)f_n(y) \dd y
\end{align*}
Thus,
\begin{align*}
    \sup_{x\in[0,1]}\int_0^1 |k(x,y)| \dd y &= \lim\limits_{n\rightarrow\infty} \int_0^1 k(x^*,y)f_n(y) \dd y \\
    &= \lim_{n\rightarrow\infty} (Tf_n)(x^*) \\
    &\leq \lim_{n\rightarrow\infty} \norm{Tf_n}_\infty
\end{align*}
But since $\norm{f_n} \leq 1$ for all $n$,
\begin{align*}
    \sup_{x\in[0,1]}\int_0^1 |k(x,y)| \dd y &\leq \lim_{n\rightarrow\infty} \norm{Tf_n}_\infty \leq \lim_{n\rightarrow\infty}\sup\limits_{\norm{f}\leq 1} \norm{Tf} = \lim_{n\rightarrow\infty}\norm{T} = \norm{T}
\end{align*}
Thus,
\begin{align*}
    \norm{T} \leq \sup\limits_{x\in[0,1]}\int_0^1 |k(x,y)|\dd y \leq \norm{T} \implies \norm{T} = \sup\limits_{x\in[0,1]}\int_0^1 |k(x,y)|\dd y
\end{align*}
\hfill $\square$

\subsection*{ b)}
\emph{Argue that the $\sup$ in \emph{(\ref{eq_2})} is attained in some $x \in [0,1]$.} \\

As argued in part \textbf{a)}, since $k$ is a continuous function of $x$ and $y$, then $\int_0^1 |k(x,y)| \dd y$ is a continuous function of $x$.  Thus $\int_0^1 |k(x,y)| \dd y$ must reach its maximum on a compact set.  Since $[0,1]$ is compact, $\exists x^* \in [0,1]$ such that $\sup\limits_{x\in[0,1]}\int_0^1 |k(x,y)|\dd y = \int_0^1 |k(x^*,y)| \dd y$.

\subsection*{ c)}
\emph{Is it possible that $\norm{T} = 1$ but $\norm{T^2} = 0$?  Prove your answer.} \\

Define $k(x,y) : [0,1] \times [0,1] \rightarrow \mathbb{R}$ as
\begin{align*}
    k(x,y) = \left\{\begin{array}{ll}
        0, & \text{if } y \leq x + \dfrac{1}{2} \\
        4(2y - 2x - 1), & \text{else}
    \end{array}\right.
\end{align*}
Then if $T$ is defined using $k$, then
\begin{align*}
    \norm{T} &= \sup\limits_{x\in[0,1]}\int_0^1|k(x,y)|\dd y \\
    &= 4\sup_{x\in[0,1]}\int_{x + \frac{1}{2}}^1 |2y - 2x - 1| \dd y \\
    &= 4\sup_{x\in[0,1]} \qty(y^2 - (2x + 1)y)\Big|_{x + \frac{1}{2}}^1 \\
    &= 4\sup_{x\in[0,1]} \qty(x - \frac{1}{2})^2 = 4\cdot\frac{1}{4} = 1
\end{align*}
However,
\begin{align*}
    \norm{T^2} &= \sup\limits_{\norm{f}=1}\norm{T^2f} \\
    &= \sup_{\norm{f}=1}\int_0^1 k(x,y)\qty(\int_0^1 k(y,s) f(s) \dd s) \dd y \\
    &= \sup_{\norm{f}=1}\int_0^1 \int_0^1 k(x,y)k(y,s) f(s) \dd s \dd y
\end{align*}
However, if $y \leq \frac{1}{2} \implies k(x,y) = 0$ and $y \geq \frac{1}{2} \implies k(y,s) = 0$, and thus $\forall y \in [0,1]$, $k(x,y)k(y,s) = 0$.  Thus,
\begin{align*}
    \norm{T^2} &= \sup_{\norm{f}=1}\int_0^1\int_0^1 k(x,y)k(y,s)f(s) \dd s \dd y \\
    &= \sup_{\norm{f}=1} \int_0^1\int_0^1 0\ \dd s \dd y = 0
\end{align*}
So it is possible for $\norm{T} = 1$ and $\norm{T^2} = 0$. \hfill $\square$

\section*{Problem 2}
\emph{Study Section 5.4 of the textbook.}

\section*{Problem 3}
\emph{Let $X$ be the Banach space $\ell^2(\mathbb{N})$, defined by}
\begin{align*}
    \ell^2(\mathbb{N}) = \left\{z = \qty(z_n)_{n=1}^\infty\ |\ z_n \in \mathbb{C},\ \sum_{n=1}^\infty |z_n|^2 < \infty\right\}
\end{align*}
\emph{For $m=1,2, \dots$, define $e_m \in X$ to be the sequence with elements $(e_m)_n = \delta_{n,m}$, and define $P_m: X \rightarrow X$ by $P_m z = z_m e_m$, for all $z \in X$.} 

\subsection*{ a)}
\emph{Prove that $P_m \in \mathcal{B}(X)$, for all $m \geq 1$.} \\

Note $\norm{e_m} = 1$ for all $m$.  Now, fix $m \geq 1$.
\begin{equation}
    \label{norm_equality}
    \norm{P_m} = \sup_{\norm{z}=1}\norm{P_mz} = \sup_{\norm{z}=1}\norm{z_me_m} = \sup_{\norm{z}=1}|z_m|\norm{e_m} = \sup_{\norm{z}=1}|z_m|
\end{equation}
Since $\norm{z} = 1$, then $|z_m| \leq 1$ for all $m$.  Thus $\sup_{\norm{z}=1}|z_m| \leq 1$, which implies $\norm{P_m} \leq 1$.  Thus $P_m$ is a bounded linear operator on $X$, i.e~$P_m \in \mathcal{B}(X)$. \hfill $\square$

\subsection*{ b)}
\emph{Verify $P_m P_n = \delta_{n,m}P_m$, for all $n,m \geq 1$.}
\begin{align*}
    (P_mP_n)(z) &= P_m(P_nz) = P_m(z_ne_n) = z_nP_m(e_n) = z_n(e_{n,m})e_m = z_n\delta_{n,m}e_m
\end{align*}
If $n \neq m$, then $\delta_{n,m} = 0$ and
\begin{align*}
    (P_mP_n)(z) = z_n\delta_{n,m}e_m = 0 = 0P_m(z) = \delta_{n,m}P_m(z) = (\delta_{n,m}P_m)(z)
\end{align*}
If $n = m$, then $z_m = z_n$ and
\begin{align*}
    (P_mP_n)(z) = z_n\delta_{n,m}e_m = \delta_{n,m}z_me_m = \delta_{n,m}P_m(z) = (\delta_{n,m}P_m)(z)
\end{align*}
In either case, $P_mP_n = \delta_{n,m}P_m$. \hfill $\square$

\subsection*{ c)}
\emph{Prove that $\norm{P_m} = 1$, for all $m \geq 1$.} \\

By part \textbf{a)}, $\norm{P_m} \leq 1$.  However,
\begin{align*}
    \norm{P_me_m} = \norm{e_{m,m}e_m} = |e_{m,m}|\norm{e_m}
\end{align*}
But $e_{m,m} = \delta_{m,m} = 1$ and $\norm{e_m} = 1$, thus $\norm{P_me_m} = 1$.  This implies $\norm{P_m} \geq 1$, and so $\norm{P_m} = 1$. \hfill $\square$

\subsection*{ d)}
\emph{For $m \geq 1$, define $S_m$ by}
\begin{align*}
    S_m = \sum_{k=1}^m P_k
\end{align*}
\emph{Calculate $S_m S_n$, for all $n,m \geq 1$.} \\

\begin{align*}
    (S_mS_n)(z) &= S_m(S_nz) \\
    &= S_m\qty(\sum_{k=1}^nP_kz) \\
    &= S_m\qty(\sum_{k=1}^nz_ke_k) \\
    &= S_m\qty(z_1,z_2, \dots, z_n, 0, 0\dots) \\
    &= \sum_{j=1}^m P_k(z_1, \dots, z_n, 0, 0, \dots)
\end{align*}
If $n \leq m$,
\begin{align*}
    (S_mS_n)(z) &= P_1(z_1, \dots, z_n, 0, 0, \dots) + \dots + P_n(z_1, \dots, z_n, 0, 0, \dots) \\
    &\ \ \ + P_{n+1}(z_1, \dots, z_n, 0, 0, \dots) + \dots + P_m(z_1, \dots, z_n, 0, 0, \dots) \\
    &= z_1e_1 + \dots + z_ne_n + 0e_{n+1} + \dots 0e_m \\
    &= (z_1, z_2, \dots, z_n) \\
    &= S_nz
\end{align*}
So $S_mS_n = S_n$.  If $n \geq m$,
\begin{align*}
    (S_mS_n)(z) &= P_1(z_1, \dots, z_n, 0, 0, \dots) + \dots + P_m(z_1, \dots, z_n, 0, 0, \dots) \\
    &= z_1e_1 + \dots + z_me_m \\
    &= (z_1, z_2, \dots, z_m) \\
    &= S_mz
\end{align*}
So $S_mS_n = S_m$.  In either case, $S_mS_n = S_{\min\{m,n\}}$. \hfill $\square$

\subsection*{ e)}
\emph{Show that $\norm{S_m} = 1$, for all $m \geq 1$.} \\

Fix $m \geq 1$.  First we show $\norm{S_m}$ is bounded by $1$.
\begin{align*}
    \norm{S_m} &= \norm{\sum_{k=1}^m P_k} \\
    &= \sup_{\norm{z}=1}\norm{\qty(\sum_{k=1}^m P_k)z} \\
    &= \sup_{\norm{z}=1}\norm{\sum_{k=1}^m P_kz} \\
    &= \sup_{\norm{z}=1}\norm{\sum_{k=1}^m z_ke_k} \\
    &= \sup_{\norm{z}=1}\norm{(z_1, z_2, \dots, z_m, 0, 0, \dots)} \\
    &\leq \sup_{\norm{z}=1}\norm{z} = 1
\end{align*}
However,
\begin{align*}
    \norm{S_me_1} &= \norm{\qty(\sum_{k=1}^mP_k)e_1} \\
    &= \norm{\sum_{k=1}^mP_ke_1} \\
    &= \norm{\sum_{k=1}^m e_{1,k}e_k} \\
    &= \norm{\sum_{k=1}^m\delta_{1,k}e_k} \\
    &= \norm{(\delta_{1,1}, \delta_{1,2}, \dots, \delta_{1,m}, 0, 0, \dots)} \\
    &= \norm{(1, 0, 0, \dots)} = 1
\end{align*}
Thus $\norm{S_m} \geq 1$, implying $\norm{S_m} = 1$ \hfill $\square$

\end{document}
